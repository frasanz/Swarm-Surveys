%% LyX 2.1.0 created this file.  For more info, see http://www.lyx.org/.
%% Do not edit unless you really know what you are doing.
\documentclass[12pt,british,spanish,a4paper,titlepage]{ociamthesis-lyx}
\usepackage[T1]{fontenc}
\usepackage[utf8]{luainputenc}
\usepackage{fancyhdr}
\pagestyle{fancy}
\setcounter{secnumdepth}{3}
\setcounter{tocdepth}{3}
\usepackage{setspace}
\usepackage[numbers]{natbib}
\onehalfspacing

\makeatletter
%%%%%%%%%%%%%%%%%%%%%%%%%%%%%% Textclass specific LaTeX commands.
\usepackage{enumitem}		% customizable list environments
\newlength{\lyxlabelwidth}      % auxiliary length 
% labeling-like list based on enumitem's description list with
% mandatory second argument (label-pattern):
\newenvironment{elabeling}[2][]%
{\settowidth{\lyxlabelwidth}{#2}
\begin{description}[font=\normalfont,style=sameline,
leftmargin=\lyxlabelwidth,#1]}
{\end{description}}

\makeatother

\usepackage{babel}
\addto\shorthandsspanish{\spanishdeactivate{~<>}}

\usepackage{listings}
\addto\captionsbritish{\renewcommand{\lstlistingname}{Listing}}
\addto\captionsspanish{\renewcommand{\lstlistingname}{Listado de código}}
\renewcommand{\lstlistingname}{Listado de código}

\begin{document}

\chapter{Enrutamiento}

El diseño de rutas de la plataforma, nos permite tener una visión
clara de la funcionalidad de la aplicación.

Como ya hemos explicado anteriormente en el esqueleto de la aplicación,
la plataforma está subdividida en varios módulos, a cada módulo se
accede mediante una ruta distinta. A la hora de elegir las rutas siempre
se ha tenido en cuenta el requisito de que sean amigables y fácilmente
recordables.

A continuación veremos como Flask resuelve las rutas, así como un
resumen del mapa de rutas de la aplicación.


\section{Resolución de rutas en Flask}

Las rutas son las URL o localizador de recursos uniforme que nos permite
acceder a las distintas funcionalidades de la plataforma.

Flask posee tres formas de definir las reglas de las rutas en el sistema:
\begin{itemize}
\item Usando el decorador \emph{flask.Flask.route()}
\item Usando la función \emph{flask.Flask.add\_url\_rule()}
\item Accediendo directamente al sistema de rutas de Werkzeug, el cual está
expuesto mediante \emph{flask.Flask.url\_map}
\end{itemize}
Para el desarrollo de la plataforma se ha elegido la opción del decorador,
debido a la sencillez de uso y a la abstracción de la función y la
ruta usada para esta.

La ruta se define mediante una cadena que simboliza la regla de una
URL, la cual puede contener variables, de tal manera que la variable
de la URL se le pasará a la función de Python que está decorando.
Además en estas rutas se puede limitar el método de petición HTTP
(GET, POST, PUT...). 

En el siguiente ejemplo vemos que para acceder a la vista que nos
permite modificar una encuesta, debemos usar la ruta /survey/N, siendo
N el número de la encuesta, además podemos acceder mediante los métodos
GET y POST. También se puede acceder mediante la cadena \textquotedbl{}titulo\_de\_la\_encuesta\_id\_survey\textquotedbl{},
para facilitar el acceso.
\begin{lstlisting}
@blueprint.route('/survey/<int:id_survey>', methods = ['GET', 'POST'])
@blueprint.route('/survey/<string:title_survey>', methods = ['GET', 'POST'])
...
def editSurvey(id_survey):
\end{lstlisting}


http://flask.pocoo.org/docs/api/\#url-route-registrations


\section{Rutas de la aplicación}

Como se explica en la sección \ref{sec:Plataforma-Modular}, la plataforma
creada tiene distintos módulos, cada una registrada en un prefijo
URL distinto, gracias al uso de \emph{blueprints}, a continuación
vamos a detallar las distintas rutas de la aplicación dividida en
módulos.


\subsection{Módulo main}

Todas las direcciones de este módulo están bajo la raíz de la dirección
web y están disponibles para todos los usuarios.
\begin{elabeling}{00.00.0000}
\item [{/}] - Raíz de la aplicación
\item [{/main}] - Raiz de la aplicación
\end{elabeling}
\begin{description}[font=\normalfont,nolistsep,leftmargin=*]
\item[Biology] Study of life.
\item[Physics] Science of matter and its motion.
\item[Psychology] Scientific study of mental processes and behaviour.
\end{description}


\subsection{Módulo de autentificación}

Todas las direcciones de este módulo están bajo la ruta \emph{/auth
}y están disponibles para todos los usuarios.
\begin{elabeling}{00.00.0000}
\item [{/}] - Autentificación mediante OpenId
\item [{/login}] - Autentificación mediante OpenId
\item [{/loginEmail}] - Autentificación mediante correo/contraseña
\item [{/register}] - Registro de cuenta de usuario
\item [{/logount}] - Cierre de la sesión actual
\end{elabeling}

\subsection{Módulo de investigador}

Todas las direcciones de este módulo están bajo la ruta\emph{ /researcher
}y solo pueden acceder aquellos usuarios que sean investigadores.
\selectlanguage{british}%
\begin{description}[nolistsep,leftmargin=*]
\item [{\foreignlanguage{spanish}{/}}] \foreignlanguage{spanish}{- Listado
de las encuestas de un investigador}
\selectlanguage{spanish}%
\item [{/index}] - Listado de las encuestas de un investigador
\item [{/new}] - Creación de una nueva encuesta
\item [{/survey/<survey>}] - Modificación de una encuesta
\item [{/survey/deleteSurvey/<survey>}] - Eliminación de una encuesta
\item [{/survey/exportSurvey/<survey>}] - Exportación de una encuesta
\item [{/survey/<survey>/consent/add}] - Creación de un consentimiento
\item [{/survey/<survey>/consent/<consent>/delete}] - Eliminación un consentimiento
\item [{/survey/<survey>/consent/<consent>}] - Modificación de un consentimiento
\item [{/survey/<survey>/section/new}] - Creación de una sección nueva
\item [{/survey/<survey>/section/<section>}] - Modificación de una sección
\item [{/survey/<survey>/deleteSection/<section>}] - Eliminación de una
sección
\item [{researcher/survey/<survey>/duplicateSection/<section>/section/<section2>}] -
Duplicación de una sección
\item [{researcher/survey/<survey>/duplicateSection/<section>/survey/}] -
Duplicar una sección
\item [{/survey/<survey>/section/<section>/new}] - Creación de una subsección
\item [{/survey/<survey>/section/<section>/addQuestion}] - Creación de
una pregunta
\item [{/survey/<survey>/section/<section>/question/<question>}] - Modificación
de una pregunta
\item [{/survey/<survey>/Section/<section>/deleteQuestion/<question>}] -
Eliminación de una pregunta
\item [{/survey/exportStats/<survey>}] - Exportación de las estadísticas
de una encuesta
\end{description}
\selectlanguage{spanish}%

\subsection{Módulo del encuestado}

Todas las direcciones de este módulo están bajo la ruta\emph{ /survey}
y está disponible para todos los usuarios que hayan iniciado sesión.
\begin{elabeling}{00.00.0000}
\item [{/}] - Listado de las encuestas
\item [{/index}] - Listado de las encuestas
\item [{/<survey>}] - Inicio de una encuesta
\item [{/<survey>/consent}] - Muestra el consentimiento de una encuesta
\item [{/<survey>/consent/<consent>}] - Muestra el consentimiento de una
encuesta
\item [{/<survey>/section/<section>}] - Muestra una sección de una encuesta
para que el encuestado la responda
\end{elabeling}

\subsection{Módulo de feedback}

Todas las direcciones de este módulo se encuentran bajo la ruta \emph{/feedback
}y está disponible para todos los usuarios que hayan iniciado sesión.
\begin{elabeling}{00.00.0000}
\item [{/<survey>}] - Muestra feedback sobre las decisiones tomadas en
la encuesta
\item [{/<survey>/<feedback>}] - Muestra feedback sobre las decisiones
tomadas en la encuesta
\end{elabeling}

\subsection{Módulo de visualización de estadísticas}

Todas las direcciones de visualización de estadísticas se encuentran
bajo la ruta \emph{/stats }y está disponible para todos los usuarios
que sean investigadores.
\begin{elabeling}{00.00.0000}
\item [{/}] - Listado de las encuestas disponibles
\item [{/index}] - Listado de las encuestas disponibles
\item [{/<survey>}] - Listado de los resultados de una encuesta
\item [{/<survey>/games}] - Listado de los juegos disponibles de una encuesta
\item [{/<survey>/games/<game>}] - Listados de los resultados de los juegos
de una encuesta\end{elabeling}

\end{document}
