
\chapter{Pruebas}

La fase de pruebas permite comprobar la correcci�n de los algoritmos
de la aplicaci�n y que cada m�dulo cumple con los requisitos iniciales
planteados.

Cada apartado de este punto se refiere a un tipo de prueba realizado,
como son los tests unitarios, los de carga y los de sistema, adem�s
de haber realizado varios tests con personas f�sicas.

Debido al ciclo de vida del proyecto, explicado en el apartado Gesti�n
de tiempo, el desarrollo ha sido un ciclo de vida evolutivo, por lo
que la automatizaci�n de pruebas se realiz� al final del proyecto,
y no en cada refinamiento del proyecto.


\section{Tests unitarios y cobertura de c�digo}

En estos tests se ha querido comprobar el perfecto funcionamiento
de todas las clases que se mapean en la base de datos, \emph{models},
as� como el correcto funcionamiento de los validadores creados para
la correcta comprobaci�n de los formularios,\emph{forms} y la clase
encargada en los emparejamientos de las decisiones, \emph{game}.

Dentro de las pruebas unitarias se realiz� pruebas de caja negra,
que sirven para verificar que el item que se est� probando, cuando
se dan las entradas apropiadas produce los resultados esperados. Siendo
este tipo de pruebas interesantes para comprobar el funcionamiento
de las interfaces.

Adem�s dentro de las pruebas unitarias, se realizo la cobertura de
c�digo, para comprobar el grado en que el c�digo fuente del programa
ha sido testeado en estas pruebas unitarias, as� como la identificaci�n
de c�digo nunca ejecutado.

Todos estos tests se integraron en la plataforma, pudiendose ejecutar
desde ella, como se puede ver en la figura \ref{fig:unit_test}

\begin{figure}[!tbh]
 	\begin{center}
\begin{listing}[style=consola, numbers=none]
$ ./manage.py test
test_answer (test_models.ModelsTestCase) ... ok
test_game_dictador (test_models.ModelsTestCase) ... ok
...
test_requerid_selectField (test_validators.ValidatorTest) ... ok

Module              statements  missing     excluded    branches    partial     coverage
app/game/game       289         73          0           96          16          74%
app/main/views      14          13          0           0           0           7%
app/models          756         60          0           157         5           97%
app/surveys/forms   166         94          4           85          10          48%
app/surveys/utiles  42          23          0           22          7           44%
Total               1267        263         4           360         38          83%
\end{listing}
		 \caption{Resultado de la ejecuci�n de los test unitarios}
 		\label{fig:unit_test}
	 \end{center}
\end{figure}

Pese a no llegar a una cobertura cercana al 100\%, si se analiza mas
detalladamente los ficheros, se puede comprobar que si que se ha testeado
las partes que se pretend�a analizar. 


\section{Pruebas de carga}

Este tipo de pruebas sirve para comprobar el rendimiento de la aplicaci�n
ante una carga de trabajo dado, con perspectiva para determinar el
comportamiento de la plataforma web ante un acceso concurrente de
cierto n�mero de usuarios.

Para realizar el acceso concurrente, se utilizo la herramienta JMeter,
que puede realizar todas las conexiones TCP que realizar�a un usuario
para contestar a la encuesta, adem�s posee de un monitor para medir
entre otros datos, los tiempos de respuesta.


\section{Pruebas de Sistema}

Para poder comprobar el correcto funcionamiento de la plataforma web,
se carga una encuesta compleja, concretamente la del experimento \emph{\textquotedbl{}�C�mo
son nuestros voluntarios?\textquotedbl{}.}

Una vez cargado dicha encuesta, se utiliza la herramienta JMeter,
para simular las acciones que realiza un voluntario al contestar una
encuesta, cometiendo todo tipo de errores posibles y asegurandonos
que los datos recibidos son los esperados.

Adem�s de la automatizaci�n de esta prueba con JMeter, distintas personas
usaron a su antojo la plataforma web y completaron el experimento
para comprobar su correcto funcionamiento.
